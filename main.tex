%% Author : Francis Muti - I63/1176/2018
%% STA 322 - Assignment
\documentclass[a4paper]{article}

%% Packages
\usepackage{amsmath, amssymb}
\usepackage[noend]{algpseudocode}
\usepackage{titlesec}
\usepackage{hyperref}
\usepackage{babel, blindtext}
\usepackage{float}
\usepackage[utf8]{inputenc}
\usepackage[T1]{fontenc}
\usepackage{mathpazo}
\usepackage{graphicx}
\usepackage{listings}
\usepackage{xparse}
\usepackage{esint} 
\usepackage{mathtools}

\usepackage{biblatex}
\addbibresource{main.bib}

\usepackage{bigintcalc}
\setcounter{secnumdepth}{4}

\titleformat{\paragraph}
{\normalfont\normalsize\bfseries}{\theparagraph}{1em}{}
\titlespacing*{\paragraph}
{0pt}{3.25ex plus 1ex minus .2ex}{1.5ex plus .2ex}

\makeatletter
\def\BFState{\State\hskip-\ALG@thislm}
\makeatother

%% Superscript command 
\newcommand{\ts}{\textsuperscript}

%% Absolute Command : abs{x} = |x|
\makeatletter
\DeclareRobustCommand{\abs}{\@ifstar\star@abs\normal@abs}
\newcommand{\star@abs}[1]{\left|#1\right|}
\newcommand{\normal@abs}[2][]{\mathopen{#1|}#2\mathclose{#1|}}
\makeatother

%% x bar
\newcommand{\xbar}[1]{\mskip.5\thinmuskip\overline{\mskip-.5\thinmuskip {#1} \mskip-.5\thinmuskip}\mskip.5\thinmuskip} 

%% Paragraph indentation
\setlength{\parindent}{0ex}


%% Insane Section : Produce primes
\makeatletter
\def\outputcode#1#2{{%
  \def\comma{\def\comma{, }}%
  \count@\@ne\@tempcntb#2\relax\@curtab#1\relax
  \@outputcode}}
\def\@outputcode{\loop\advance\count@\@ne
\expandafter\ifx\csname p-\the\count@\endcsname\relax
\ifnum\@tempcntb<\count@\else
  \ifnum\count@<\@curtab\else\comma\the\count@\fi\fi\else\repeat
\@tempcnta\count@\loop\advance\@tempcnta\count@
\expandafter\let\csname p-\the\@tempcnta\endcsname\@ne
\ifnum\@tempcnta<\@tempcntb\repeat
\ifnum\@tempcntb>\count@\expandafter\@outputcode\fi}
\makeatother

%% Diabolical : Ackermann function
%% Recursive function
\def\afterfi#1#2\fi{\fi#1}
\def\Ac#1#2{\ifnum#1=0 \afterfi{\the\numexpr#2+1\relax}%
            \else \afterfi{\ifnum#2=0 \afterfi{\Aeval{#1-1}{1}}%
                           \else \afterfi{\Aeval{#1-1}{\Aeval{#1}{#2-1}}}\fi}\fi}
\def\Aeval#1#2{\expanded{\noexpand\Ac{\the\numexpr#1}{\the\numexpr#2}}}
\def\ackermann(#1,#2){$ackermann(#1,#2)=\Ac{#1}{#2}$\par}

\begin{document}
\begin{titlepage}
	\begin{center}
		\vspace*{1cm}
		\Huge
		\textbf{STA 322 - Assignment}
		
		\vspace{0.5cm}		
		\LARGE
		\LaTeX \;Document
		\vspace{1.5cm}
		
		\textbf{Francis Muti} \textsc{- I63/1176/2018}
		\vfill
		
		A \LaTeX \;document to showcase scientific writing and
		use of mathematical symbols.
		
		\vspace{0.8cm}
		\includegraphics[width=0.4\textwidth]{uon.jpg}
		
		\Large
		School of Mathematics\\
		University of Nairobi\\
		\today \\
	\end{center}
\end{titlepage}

\section{Latex Exercise}
\label{sec:exercise}
	\subsection{Easy}
	%\addtolength{\jot}{1em}
	\begin{gather}
		\text{Please type me! \;The quick brown fox jumps over 
		the lazy dog} \\
		e ^{i\pi} + 1 =  0\\
		e ^{i\theta} = \cos\theta + i\sin\theta\\
		G_{\mu v} + \Lambda g_{\mu v} = \frac{8 \pi G}{c ^4} T_{\mu v}\\
		x = \frac{-b \pm \sqrt{b ^2 - 4ac}}{2a}\\
		\vec{L} = \vec{r} \times \vec{p}\\
		\sqrt[3]{2}\\
		(x + y) ^n = \sum_{r=0} ^n \binom{n}{r} x ^r y ^{n-r}\\
		\sqrt{\frac{a_1 ^2 + \dots + a_n ^2}{n}} \geq 
		\frac{a_1 + \dots + a_n}{n} \geq
		\sqrt[n]{a_1\dots a_n} \geq 
		\frac{n}{\frac{1}{a_1} + \dots + \frac{1}{a_n}}\\
		\abs{\langle x, y \rangle} ^2 
		\leq \langle x, x \rangle \cdot \langle y, y \rangle\\
		\begin{split}	
			&\text{\textbf{A1:}} \; \varphi \longrightarrow 
			(\psi \rightarrow \varphi)\\
			&\text{\textbf{A2:}} \; (\varphi \rightarrow 
			(\psi \rightarrow \theta)) \longrightarrow
			((\psi \rightarrow \varphi) \rightarrow (\phi \rightarrow
			\theta))\\
			&\text{\textbf{A3:}} \; (\neg \varphi \rightarrow
			\neg \psi) \longrightarrow (\psi \rightarrow \varphi)
        \end{split}
	\end{gather}
	
	\newpage

	\subsection{Medium}
	\begin{gather}	
	1_A =
	\begin{cases} 
		1 &\text{if $x\in A$}\\ 
		0 &\text{if $x\notin A$}\\ 
	\end{cases}\\
	n \underbrace{\uparrow \dots \uparrow}_n n = 
	n \rightarrow n \rightarrow n
	\end{gather}
	
	In the following, not the spacing between the $=$ and the
	${\ts 1} 1$, ${\ts 2} 2$, and ${\ts {\ts 3}} {\ts 3} 3$
	\begin{gather}
	\begin{split}
	1 \uparrow 1 = {\ts 1} 1 &= 1\\
	2 \uparrow \uparrow 2 = {\ts 2} 2 &= 4\\
	3 \uparrow \uparrow \uparrow 3 = {\ts {\ts 3}} {\ts 3} 3 &=
	3 \uparrow \uparrow 3 \uparrow \uparrow 3 =
	\underbrace{3 ^{3 ^{3 ^{3 ^{3 ^{3 ^{\cdot ^{ \cdot ^{\cdot ^{3}}}}}}}}}}
	_{3 ^{3 ^{3}} \text{threes}}
	\end{split}\\
	\frac{\text{d}}{\text{d}x} = \lim_{\Delta x \rightarrow 0}
	\frac{f(x + \Delta x) - f(x)}{\Delta x}\\
	\text{H}_2 \text{O}(\ell) + \text{H}_2 \text{O}(\ell)
	\leftrightharpoons \text{H}_3 \text{O} ^{+} (aq) +
	\text{OH} ^{-} (aq) \\
	\Gamma (n+1) \stackrel{def}{=} 
	\int_0 ^{\infty} e ^{-t}t ^n \text{d}t\\
	\text{$\gcd$($n, m$ mod $n$);} \quad x \equiv y \quad \text{(mod $b$);}
	\; x \equiv y \quad \text{(mod $c$); } \quad x \equiv y \quad (d)
	\end{gather}\\
	
	In the following, note the bold symbols.
	\begin{gather}
		\begin{split}
			\nabla \cdot \mathbf{E} &= \frac{\rho}{\varepsilon_0}\\
			\nabla \cdot \mathbf{B} &= 0\\
			\nabla \times \mathbf{E} &= 
			-\frac{\partial \mathbf{B}}{\partial t}\\
			\nabla \times \mathbf{B} &= \mu_0\text{\bf{J}} +
			\mu_0\varepsilon_0 \frac{\partial \mathbf{B}}{\partial t}
		\end{split}
	\end{gather}
	
	For the following exercise, you will need to use 
	\texttt{\textbackslash usepackage \{esint\}}
	to get the symbol $\oiint$.
	
	\begin{gather}
		\begin{split}
			\oiint_{\partial V} \mathbf{E} \cdot \text{d \bf{A}} &=
			\frac{\mathcal{Q}(V)}{\varepsilon_0}\\
			\oiint_{\partial V} \mathbf{B} \cdot \mathbf{A} &= 0\\
			\oint_{\partial S}  \mathbf{E}\cdot \text{d}\mathbf{l} &=
			-\frac{\partial \Phi_{B, S}}{\partial t}\\
			\oint_{\partial S} \mathbf{B} \cdot \text{d} \mathbf{l} &=
			\mu_0 {\mathbf{I}}_S + \mu_0 \varepsilon_0
			 \frac{\partial \Phi_{B, S}}{\partial t}
		\end{split}
	\end{gather}
	
	You might find the environment \texttt{bmatrix} and \texttt{pmatrix}
	useful for the following exercises.
	\begin{gather}
	\rho\theta = 
		\begin{pmatrix}
			\cos \theta & \sin \theta \\
			-\sin \theta & \cos \theta
		\end{pmatrix}
		=
		\begin{bmatrix}
			\cos \theta & \sin \theta \\
			-\sin \theta & \cos \theta
		\end{bmatrix}\\
	\begin{bmatrix}
	\begin{tabular}{c | c c c}
		1 & 0 & \text{$\cdots$} & 0\\
		\hline
		0 & \text{$\ast$} & \text{$\cdots$} & \text{$\ast$}\\
		\vdots & \vdots & \text{$\ddots$} & \vdots\\
		0 & \text{$\ast$} & \text{$\cdots$} & \text{$\ast$}
	\end{tabular}
	\end{bmatrix}
	=
	\begin{tabular}{|c | c c c|}
		\hline
		1 & 0 & \text{$\cdots$} & 0\\
		\hline
		0 & \text{$\ast$} & \text{$\cdots$} & \text{$\ast$}\\
		\vdots & \vdots & \text{$\ddots$} & \vdots\\
		0 & \text{$\ast$} & \text{$\cdots$} & \text{$\ast$}\\
		\hline
	\end{tabular}
	\end{gather}
	
	Note the locations of the bounds on the summation in the following
	exercise.
	
	\begin{gather}
		\sigma = \sqrt{\frac{1}{N}
		 \sum\limits_{i=1} ^N {p_i(x_i - \xbar{x}) ^2}}= 
		 \sqrt{\dfrac{\sum\limits_{i=1}^{N} 
		 {p_i(x_i - \xbar{x}) ^2}}{N}}\\
		\varphi(n) = n \cdot 
		\prod\limits_{\stackrel{p|n}{p\; \text{prime}}} (1 - \frac{1}{p})
	\end{gather}
	
	If you \texttt{\textbackslash usepackage \{mathtools\}}, you 
	can make it look like
	\begin{gather}
		\varphi(n) = n \cdot 
		\prod\limits_{\stackrel{p|n}{p\; \text{prime}}} 
		(1 - \frac{1}{p})\\
		\prescript{4}{12}{\mathbf{C}}\prescript{5+}{2}{}\quad
		\prescript{14}{2}{\mathbf{C}}\prescript{5+}{2}{}\quad
		\prescript{4}{12}{\mathbf{C}}\prescript{5+}{2}{}\quad
		\prescript{14}{}{\mathbf{C}}\prescript{5+}{2}{}\quad
		\prescript{}{12}{\mathbf{C}}\prescript{5+}{2}{}\quad
	\end{gather}
	
	In the following, note the size of /, and the spacing
	on the sizes of the |.
	
	\begin{gather}
		\begin{split}
		&\mathbb{Q} \cong \bigg 
		\{ \frac{a}{b} \bigg{|}\; a, b \in \mathbb{Z} 
		\text{\;and\;} b \neq 0 \bigg{\}} \;\ \bigg {/ \sim}\\
		&\frac{a}{b} \sim \frac{c}{d} \Longleftrightarrow ad - bc = 0
		\end{split}
	\end{gather}
	
	Notice both the horizontal and vertical spacing in the following 
	exercise. 
	%% (TODO: the vertical spacing issue)
	
	\begin{gather}
	\begin{split}
	1 \uparrow 1 = {\ts 1} 1 &= 1\\
	2 \uparrow \uparrow 2 = {\ts 2} 2 &= 4\\[1pt]
	3 \uparrow \uparrow \uparrow 3 = {\ts {\ts 3}} {\ts 3} 3 &=
	3 \uparrow \uparrow 3 \uparrow \uparrow 3 =
	\underbrace{3 ^{3 ^{3 ^{3 ^{3 ^{3 ^{\cdot ^{ \cdot ^{\cdot ^{3}}}}}}}}}}
	_{3 ^{3 ^{3}} \text{threes}}
	\end{split}
	\end{gather}
	
	\newpage
	\subsection{Insane}
	Write a command \texttt{outputcode}
	 which outputs the code of the document being typeset.\\\\
	%% Source : https://tex.stackexchange.com/questions/134305/how-to-produce-a-list-of-prime-numbers-in-latex
	Primes (1 - 10):\;\;\;\outputcode{1}{10}\\
	Primes (1 - 20):\;\;\;\outputcode{1}{20}\\
	Primes (50 - 100):\;\outputcode{50}{75}
	
	\subsection{Diabolical}
	The Ackermann function is defined as \cite{latex}
	\begin{gather*}	
		A(m, n) = \begin{cases}
			n+1 & \text{if $m = 0$}\\
			A(m-1, 1) & \text{if $m > 0$ \;and\; $n=0$}\\
			A(m-1, A(m, n-1)) & \text{if $m > 0$ and $n > 0$}
		\end{cases}
	\end{gather*}
	Only outputs the result.\\
	\ackermann(2, 2)
	\ackermann(1, 2)
	
	
\printbibheading
\printbibliography
\end{document}


























