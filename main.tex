%% Author : Francis Muti - I63/1176/2018
%% STA 322 - Assignment
\documentclass[a4paper]{article}

%% Packages
\usepackage{amsmath, amssymb}
\usepackage[noend]{algpseudocode}
\usepackage{titlesec}
\usepackage{hyperref}
\usepackage{babel, blindtext}
\usepackage{float}
\usepackage[utf8]{inputenc}
\usepackage[T1]{fontenc}
\usepackage{mathpazo}
\usepackage{graphicx}
\usepackage{listings}
\usepackage{xparse}
\usepackage{esint} 

\setcounter{secnumdepth}{4}

\titleformat{\paragraph}
{\normalfont\normalsize\bfseries}{\theparagraph}{1em}{}
\titlespacing*{\paragraph}
{0pt}{3.25ex plus 1ex minus .2ex}{1.5ex plus .2ex}

\makeatletter
\def\BFState{\State\hskip-\ALG@thislm}
\makeatother

%% Superscript command 
\newcommand{\ts}{\textsuperscript}

%% Absolute Command : abs{x} = |x|
\makeatletter
\DeclareRobustCommand{\abs}{\@ifstar\star@abs\normal@abs}
\newcommand{\star@abs}[1]{\left|#1\right|}
\newcommand{\normal@abs}[2][]{\mathopen{#1|}#2\mathclose{#1|}}
\makeatother

% Paragraph indentation
\setlength{\parindent}{0ex}
\begin{document}
\begin{titlepage}
	\begin{center}
		\vspace*{1cm}
		\Huge
		\textbf{STA 322 - Assignment}
		
		\vspace{0.5cm}		
		\LARGE
		\LaTeX \;Document
		\vspace{1.5cm}
		
		\textbf{Francis Muti} \textsc{- I63/1176/2018}
		\vfill
		
		A \LaTeX \;document to showcase scientific writing and
		use of mathematical symbols.
		
		\vspace{0.8cm}
		\includegraphics[width=0.4\textwidth]{uon.jpg}
		
		\Large
		School of Mathematics\\
		University of Nairobi\\
		\today \\
	\end{center}
\end{titlepage}

\section{Latex Exercise}
\label{sec:exercise}
	\subsection{Easy}
	%\addtolength{\jot}{1em}
	\begin{gather}
		\text{Please type me! \;The quick brown fox jumps over 
		the lazy dog} \\
		e ^{i\pi} + 1 =  0\\
		e ^{i\theta} = \cos\theta + i\sin\theta\\
		G_{\mu v} + \Lambda g_{\mu v} = \frac{8 \pi G}{c ^4} T_{\mu v}\\
		x = \frac{-b \pm \sqrt{b ^2 - 4ac}}{2a}\\
		\vec{L} = \vec{r} \times \vec{p}\\
		\sqrt[3]{2}\\
		(x + y) ^n = \sum_{r=0} ^n \binom{n}{r} x ^r y ^{n-r}\\
		\sqrt{\frac{a_1 ^2 + \dots + a_n ^2}{n}} \geq 
		\frac{a_1 + \dots + a_n}{n} \geq
		\sqrt[n]{a_1\dots a_n} \geq 
		\frac{n}{\frac{1}{a_1} + \dots + \frac{1}{a_n}}\\
		\abs{\langle x, y \rangle} ^2 
		\leq \langle x, x \rangle \cdot \langle y, y \rangle\\
		\begin{split}	
			&\text{\textbf{A1:}} \; \varphi \longrightarrow 
			(\psi \rightarrow \varphi)\\
			&\text{\textbf{A2:}} \; (\varphi \rightarrow 
			(\psi \rightarrow \theta)) \longrightarrow
			((\psi \rightarrow \varphi) \rightarrow (\phi \rightarrow
			\theta))\\
			&\text{\textbf{A3:}} \; (\neg \varphi \rightarrow
			\neg \psi) \longrightarrow (\psi \rightarrow \varphi)
        \end{split}
	\end{gather}
	
	\newpage

	\subsection{Medium}
	\begin{gather}	
	1_A =
	\begin{cases} 
		1 &\text{if $x\in A$}\\ 
		0 &\text{if $x\notin A$}\\ 
	\end{cases}\\
	n \underbrace{\uparrow \dots \uparrow}_n n = 
	n \rightarrow n \rightarrow n
	\end{gather}
	
	In the following, not the spacing between the $=$ and the
	${\ts 1} 1$, ${\ts 2} 2$, and ${\ts {\ts 3}} {\ts 3} 3$
	\begin{gather}
	\begin{split}
	1 \uparrow 1 = {\ts 1} 1 &= 1\\
	2 \uparrow \uparrow 2 = {\ts 2} 2 &= 4\\
	3 \uparrow \uparrow \uparrow 3 = {\ts {\ts 3}} {\ts 3} 3 &=
	3 \uparrow \uparrow 3 \uparrow \uparrow 3 =
	\underbrace{3 ^{3 ^{3 ^{3 ^{3 ^{3 ^{\cdot ^{ \cdot ^{\cdot ^{3}}}}}}}}}}
	_{3 ^{3 ^{3}} \text{threes}}
	\end{split}\\
	\frac{\text{d}}{\text{d}x} = \lim_{\Delta x \rightarrow 0}
	\frac{f(x + \Delta x) - f(x)}{\Delta x}\\
	\text{H}_2 \text{O}(\ell) + \text{H}_2 \text{O}(\ell)
	\leftrightharpoons \text{H}_3 \text{O} ^{+} (aq) +
	\text{OH} ^{-} (aq) \\
	\Gamma (n+1) \stackrel{def}{=} 
	\int_0 ^{\infty} e ^{-t}t ^n \text{d}t\\
	\text{gcd($n, m$ mod $n$);} \quad x \equiv y \quad \text{(mod $b$);}
	\; x \equiv y \quad \text{(mod $c$); } \quad x \equiv y \quad (d)
	\end{gather}\\
	
	In the following, note the bold symbols.
	\begin{gather}
		\begin{split}
			\nabla \cdot \text{\bf{E}} &= \frac{\rho}{\varepsilon_0}\\
			\nabla \cdot \text{\bf{B}} &= 0\\
			\nabla \times \text{\bf{E}} &= 
			-\frac{\partial \text{\bf{B}}}{\partial t}\\
			\nabla \times \text{\bf{B}} &= \mu_0\text{\bf{J}} +
			\mu_0\varepsilon_0 \frac{\partial \text{\bf{B}}}{\partial t}
		\end{split}
	\end{gather}
	
	For the following exercise, you will need to use 
	\texttt{\textbackslash usepackage \{esint\}}
	to get the symbol $\oiint$.
	
	\begin{gather}
		\begin{split}
			\oiint_{\partial V} \text{\bf{E}} \cdot \text{d \bf{A}} &=
			\frac{\mathcal{Q}(V)}{\varepsilon_0}\\
			\oiint_{\partial V} \text{\bf{B}} \cdot \text{d \bf{A}} &= 0\\
			\oint_{\partial S}  \text{\bf{E}} \cdot \text{d \bf{l}} &=
			-\frac{\partial \Phi_{B, S}}{\partial t}\\
			\oint_{\partial S} \text{\bf{B}} \cdot \text{d \bf{l}} &=
			\mu_0 {\bf{I}}_S + \mu_0 \varepsilon_0
			 \frac{\partial \Phi_{B, S}}{\partial t}
		\end{split}
	\end{gather}
	
	You might find the environment \texttt{bmatrix} and \texttt{pmatrix}
	useful for the following exercises.
	\begin{gather}
	\rho\theta = 
		\begin{pmatrix}
			\cos \theta & \sin \theta \\
			-\sin \theta & \cos \theta
		\end{pmatrix}
		=
		\begin{bmatrix}
			\cos \theta & \sin \theta \\
			-\sin \theta & \cos \theta
		\end{bmatrix}\\
	\begin{bmatrix}
	\begin{tabular}{c | c c c}
		1 & 0 & \text{$\cdots$} & 0\\
		\hline
		0 & \text{$\ast$} & \text{$\cdots$} & \text{$\ast$}\\
		\vdots & \vdots & \text{$\ddots$} & \vdots\\
		0 & \text{$\ast$} & \text{$\cdots$} & \text{$\ast$}
	\end{tabular}
	\end{bmatrix}
	=
	\begin{tabular}{|c | c c c|}
		\hline
		1 & 0 & \text{$\cdots$} & 0\\
		\hline
		0 & \text{$\ast$} & \text{$\cdots$} & \text{$\ast$}\\
		\vdots & \vdots & \text{$\ddots$} & \vdots\\
		0 & \text{$\ast$} & \text{$\cdots$} & \text{$\ast$}\\
		\hline
	\end{tabular}
	\end{gather}
	
	Note the locations of the bounds
\end{document}


























